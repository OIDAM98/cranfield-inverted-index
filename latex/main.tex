%%%%%%%%%%%%%%%%%%%%%%%%%%%%%%%%%%%%%%%%%
% Lachaise Assignment
% LaTeX Template
% Version 1.0 (26/6/2018)
%
% This template originates from:
% http://www.LaTeXTemplates.com
%
% Authors:
% Marion Lachaise & François Févotte
% Vel (vel@LaTeXTemplates.com)
%
% License:
% CC BY-NC-SA 3.0 (http://creativecommons.org/licenses/by-nc-sa/3.0/)
% 
%%%%%%%%%%%%%%%%%%%%%%%%%%%%%%%%%%%%%%%%%

%----------------------------------------------------------------------------------------
%	PACKAGES AND OTHER DOCUMENT CONFIGURATIONS
%----------------------------------------------------------------------------------------

\documentclass{article}

%%%%%%%%%%%%%%%%%%%%%%%%%%%%%%%%%%%%%%%%%
% Lachaise Assignment
% Structure Specification File
% Version 1.0 (26/6/2018)
%
% This template originates from:
% http://www.LaTeXTemplates.com
%
% Authors:
% Marion Lachaise & François Févotte
% Vel (vel@LaTeXTemplates.com)
%
% License:
% CC BY-NC-SA 3.0 (http://creativecommons.org/licenses/by-nc-sa/3.0/)
% 
%%%%%%%%%%%%%%%%%%%%%%%%%%%%%%%%%%%%%%%%%

%----------------------------------------------------------------------------------------
%	PACKAGES AND OTHER DOCUMENT CONFIGURATIONS
%----------------------------------------------------------------------------------------

\usepackage{amsmath,amsfonts,stmaryrd,amssymb} % Math packages

\usepackage{enumerate} % Custom item numbers for enumerations

\usepackage[ruled]{algorithm2e} % Algorithms

\usepackage[framemethod=tikz]{mdframed} % Allows defining custom boxed/framed environments

\usepackage{listings} % File listings, with syntax highlighting
\lstset{
	basicstyle=\ttfamily, % Typeset listings in monospace font
}

%----------------------------------------------------------------------------------------
%	DOCUMENT MARGINS
%----------------------------------------------------------------------------------------

\usepackage{geometry} % Required for adjusting page dimensions and margins

\geometry{
	paper=a4paper, % Paper size, change to letterpaper for US letter size
	top=2.5cm, % Top margin
	bottom=3cm, % Bottom margin
	left=2.5cm, % Left margin
	right=2.5cm, % Right margin
	headheight=14pt, % Header height
	footskip=1.5cm, % Space from the bottom margin to the baseline of the footer
	headsep=1.2cm, % Space from the top margin to the baseline of the header
	%showframe, % Uncomment to show how the type block is set on the page
}

%----------------------------------------------------------------------------------------
%	FONTS
%----------------------------------------------------------------------------------------

\usepackage[utf8]{inputenc} % Required for inputting international characters
\usepackage[T1]{fontenc} % Output font encoding for international characters

\usepackage{XCharter} % Use the XCharter fonts

%----------------------------------------------------------------------------------------
%	COMMAND LINE ENVIRONMENT
%----------------------------------------------------------------------------------------

% Usage:
% \begin{commandline}
%	\begin{verbatim}
%		$ ls
%		
%		Applications	Desktop	...
%	\end{verbatim}
% \end{commandline}

\mdfdefinestyle{commandline}{
	leftmargin=10pt,
	rightmargin=10pt,
	innerleftmargin=15pt,
	middlelinecolor=black!50!white,
	middlelinewidth=2pt,
	frametitlerule=false,
	backgroundcolor=black!5!white,
	frametitle={Command Line},
	frametitlefont={\normalfont\sffamily\color{white}\hspace{-1em}},
	frametitlebackgroundcolor=black!50!white,
	nobreak,
}

% Define a custom environment for command-line snapshots
\newenvironment{commandline}{
	\medskip
	\begin{mdframed}[style=commandline]
}{
	\end{mdframed}
	\medskip
}

%----------------------------------------------------------------------------------------
%	FILE CONTENTS ENVIRONMENT
%----------------------------------------------------------------------------------------

% Usage:
% \begin{file}[optional filename, defaults to "File"]
%	File contents, for example, with a listings environment
% \end{file}

\mdfdefinestyle{file}{
	innertopmargin=1.6\baselineskip,
	innerbottommargin=0.8\baselineskip,
	topline=false, bottomline=false,
	leftline=false, rightline=false,
	leftmargin=2cm,
	rightmargin=2cm,
	singleextra={%
		\draw[fill=black!10!white](P)++(0,-1.2em)rectangle(P-|O);
		\node[anchor=north west]
		at(P-|O){\ttfamily\mdfilename};
		%
		\def\l{3em}
		\draw(O-|P)++(-\l,0)--++(\l,\l)--(P)--(P-|O)--(O)--cycle;
		\draw(O-|P)++(-\l,0)--++(0,\l)--++(\l,0);
	},
	nobreak,
}

% Define a custom environment for file contents
\newenvironment{file}[1][File]{ % Set the default filename to "File"
	\medskip
	\newcommand{\mdfilename}{#1}
	\begin{mdframed}[style=file]
}{
	\end{mdframed}
	\medskip
}

%----------------------------------------------------------------------------------------
%	NUMBERED QUESTIONS ENVIRONMENT
%----------------------------------------------------------------------------------------

% Usage:
% \begin{question}[optional title]
%	Question contents
% \end{question}

\mdfdefinestyle{question}{
	innertopmargin=1.2\baselineskip,
	innerbottommargin=0.8\baselineskip,
	roundcorner=5pt,
	nobreak,
	singleextra={%
		\draw(P-|O)node[xshift=1em,anchor=west,fill=white,draw,rounded corners=5pt]{%
		Question \theQuestion\questionTitle};
	},
}

\newcounter{Question} % Stores the current question number that gets iterated with each new question

% Define a custom environment for numbered questions
\newenvironment{question}[1][\unskip]{
	\bigskip
	\stepcounter{Question}
	\newcommand{\questionTitle}{~#1}
	\begin{mdframed}[style=question]
}{
	\end{mdframed}
	\medskip
}

%----------------------------------------------------------------------------------------
%	WARNING TEXT ENVIRONMENT
%----------------------------------------------------------------------------------------

% Usage:
% \begin{warn}[optional title, defaults to "Warning:"]
%	Contents
% \end{warn}

\mdfdefinestyle{warning}{
	topline=false, bottomline=false,
	leftline=false, rightline=false,
	nobreak,
	singleextra={%
		\draw(P-|O)++(-0.5em,0)node(tmp1){};
		\draw(P-|O)++(0.5em,0)node(tmp2){};
		\fill[black,rotate around={45:(P-|O)}](tmp1)rectangle(tmp2);
		\node at(P-|O){\color{white}\scriptsize\bf !};
		\draw[very thick](P-|O)++(0,-1em)--(O);%--(O-|P);
	}
}

% Define a custom environment for warning text
\newenvironment{warn}[1][Warning:]{ % Set the default warning to "Warning:"
	\medskip
	\begin{mdframed}[style=warning]
		\noindent{\textbf{#1}}
}{
	\end{mdframed}
}

%----------------------------------------------------------------------------------------
%	INFORMATION ENVIRONMENT
%----------------------------------------------------------------------------------------

% Usage:
% \begin{info}[optional title, defaults to "Info:"]
% 	contents
% 	\end{info}

\mdfdefinestyle{info}{%
	topline=false, bottomline=false,
	leftline=false, rightline=false,
	nobreak,
	singleextra={%
		\fill[black](P-|O)circle[radius=0.4em];
		\node at(P-|O){\color{white}\scriptsize\bf i};
		\draw[very thick](P-|O)++(0,-0.8em)--(O);%--(O-|P);
	}
}

% Define a custom environment for information
\newenvironment{info}[1][Info:]{ % Set the default title to "Info:"
	\medskip
	\begin{mdframed}[style=info]
		\noindent{\textbf{#1}}
}{
	\end{mdframed}
}
 % Include the file specifying the document structure and custom commands

%----------------------------------------------------------------------------------------
%	ASSIGNMENT INFORMATION
%----------------------------------------------------------------------------------------

\title{LIS4031: Project \#1} % Title of the assignment
\title{Boolean Retrieval Model} % Title of the assignment

\author{
	Dulio Paolo Caggiano Amand\\ 
	\texttt{dulio.caggianoad@udlap.mx} 
	\and
	Oscar Iv\'an de Alva Mart\'inez\\ 
	\texttt{oscar.dealvamz@udlap.mx}
} % Author name and email address

\date{Universidad de las Am\'ericas  Puebla --- \today} % University, school and/or department name(s) and a date

%----------------------------------------------------------------------------------------

\begin{document}

\maketitle % Print the title

%----------------------------------------------------------------------------------------
%	INTRODUCTION
%----------------------------------------------------------------------------------------

\section*{Introduction} % Unnumbered section

In this practice, research and development focused on the \textit{Boolean Information Retrieval Model} was carried out to define the basic concepts of a retrieval model. For the development of this project, a dictionary and a posting list were used in order to implement an Inverted iIdex based on the Cranfield collection.

For the implementation of this project, the programming language \textit{Scala} was used as it provides a concise notation that grants the creation of a \textit{Domain Specific Language} (DSL) for the ability to model and create a solution for the problem. Also, \textit{Scala} has a tight compatibility and interoperability with \textit{Java}, that allows to take advantage of both the \textit{Java Virtual Machine} (JVM) and the libraries of \textit{Java}, and at the same time having its own libraries. Furthermore, its strict type system has the advantage of allowing the creation of specific types that model the solution of this problem.

%----------------------------------------------------------------------------------------
%	PROBLEM 1
%----------------------------------------------------------------------------------------

\section{Boolean Retrieval Model} % Numbered section

The Cranfield collection is based on three main components, the first one is a collection of documents which contains information from different investigations. The second component is the one that contains a set of specific queries for the aforementioned data collection and the last component contains the relevance judgments which are all the factors that are taken in consideration to categorize a document as relevant.

The documents contained in the Cranfield collection generally maintain a structure, where the information contained in the document is specified by means of headings, for example, the usage of the header \textit{.I} to describe the documents ID or \textit{.A} to list the authors of the document.

%------------------------------------------------

\subsection{Parsing the collection}

To transform the words found in the documents, a dedicated parser was created to perform the tokenization of these. This parser has the functionality of creating tokens from the headers established within each document, which allows establishing metadata, in the form of classes, for the words found.

Also, specific classes were created for the documentation of words and metadata, and a class was established to store the content of a document. These classes represent the different types of header found within the document, these are: document number, author, title, affiliation, and abstract text. The established classes are the following:

\begin{lstlisting}[language=Scala]
sealed trait Header
case class Id(id: Int) extends Header
case class Title(text: String) extends Header
case class Author(fullName: String) extends Header
case class Adscription(adscription: String) extends Header
case class Abstract(text: String) extends Header
case class Document(tags: List[Header])
	\end{lstlisting}

On the other hand, the internal structure of a document was established as a List of parsed headers. This decision was made because not all of the documents within the collection have a defined structure, which makes it difficult to generalize and represent them with only one.

In order to parse the words, the \textbf{cran.all.1400} file was first divided into its documents, which are specified at the beginning with the \textit{.I} header and followed by the document number. This was done in order to allow an easier identification of the words inside each document.

Then, each document was divided into the sections previously mentioned, which are identified by the headings: \textit{.A}, \textit{.T}, \textit{.B} and \textit{.W}, along with the document number where they are located. Finally, these elements were stored within a \textbf{Document} class to indicate that a document from the collection was parsed along with its contents.

%------------------------------------------------

\subsection{Inverted Index}

The \textit{Inverted Index} was implemented by using a \textit{Dictionary} which gave mainly the advantage of fast access time once you search for a word. The keys used in the dictionary were all the different words found in the documents and the values are stored inside a \textit{Tuple}, with the first element representing the number of occurrences that the word has in all the processed documents and the second element is a \textit{Set} which contains the IDs of the documents in which the word is found. The \textit{Set} data structure is used to avoid the repetition of IDs, taking advantage of this characteristic.

Once the whole file is parsed and the words are tokenized, the list of tokens is processed element by element, adding them to the \textit{Inverted Index}. When adding an element to the \textit{Inverted index} and it is already stored inside of it, then the ocurrence counter is increased and the document id where it appeared is added to the \textit{Set}.

\subsection{Parsing the query}

To parse the query that the user enters in the program, a free context grammar was used, since the structure of a query was already defined. This grammar is based upon the grammar used to solve basic algebraic operations. This decision was made because the operators dealing with the \textit{Boolean Retrievl Model} work the same as the basic mathematical operators of \textbf{+}, \textbf{*} and \textbf{- \textit{of a scalar}}. The grammar for a query is defined as follows:

\begin{align*}
	Expr   & ::= Term (\text{'OR'}  Term)*                          \\
	Term   & ::= Factor (\text{'AND'} Factor)*                      \\
	Factor & ::= [\text{'NOT'}] (Word | \text{'('} Expr \text{')'}) \\
	Word   & ::= [a-zA-Z]
\end{align*}

This definition allows the parser to be able to analyze a query through the precedence of the \textbf{AND}, \textbf{OR} and \textbf{NOT} operators, as well as the parentheses. To make the parsing of the query, the Scala Parser Combinators library was used, since it allows to represent a grammar free of context in code and define the procedurements while parsing. This library uses its own DSL to both model the grammar and the Succes case if the expression is a match. The following code represents the way to express the \textit{Expr} case of the grammar:

\begin{lstlisting}[language=Scala]
def expr: Parser[BinaryArray] = term ~ rep("or" ~> term) ^^ {
    case expr ~ list =>
      list.foldLeft(expr) {
        case (x, y) => x | y // Make the OR operator of the terms
      }
  }
		\end{lstlisting}

The \textit{CranParser} is in charge of identifying the specified grammar cases and asking the \textit{Boolean Retrieval Model} to carry out a specified operation. When the parser finds a word in query, it calls the \textbf{convertToBinaryArray(word)} method of the \textit{Boolean Retrieval Model} in order to tokenize the word based upon the following method:

\begin{lstlisting}[language=Scala]
private def word: Parser[BinaryArray] =
    """[a-zA-z]+""".r ^^ (brm.convertToBinaryArray(_))
			\end{lstlisting}

\subsection{Boolean Retrieval Model}

When the words are parsed inside the query, they are transformed into a \textit{BinaryArray} class, which is an \textit{Array} of \textit{Shorts} of the size of the number of processed documents, being 1400 for the Cranfield documents. The \textit{BinaryArray} limits the data that can be used in the array to 0 and 1 in order to represent the ocurrence of a word inside the different documents in a binary way. This representation consists of a 1 in the position \textbf{i} if the element is in the \textbf{document i} and if it is not found inside \textbf{document i} then will be represented with a 0. The following method is in charge to convert a word to its \textit{BinaryArray} representation:

\begin{lstlisting}[language=Scala]
def convertToBinaryArray(word: String) = {
    index.get(word) match {
      case Some(value) =>
        value match {
          case (_, documents) =>
            documents
              .map(_.id)
              .foldLeft(BinaryArray.fill(index.numberOfDocs)(0)) {                
                case (acc, index) => acc.update(index - 1, 1)
              }
        }

      case None => BinaryArray.fill(index.numberOfDocs)(0)
    }
  }
\end{lstlisting}

The \textbf{convertToBinaryArray(word)} method is in charge to convert a word to a \textit{BinaryArray} making the following steps:

\begin{enumerate}[(i)]
	\item Get the word from the \textit{Inverted Index}.
	\begin{enumerate}[(i)]
		\item If the word is a key in the index:
	\begin{enumerate}[(a)]
		\item Get its documents where it appears, its \textit{Set[Id]}.
		\item Generate a new \textit{BinaryArray} that is empty (all its indexes are 0).
		\item Start filling this array with 1's in the index where a word appears.
	\end{enumerate}
	\item If its not in the index:
	\begin{enumerate}[(a)]
		\item Return an empty \textit{BinaryArray}, since it is not in any document.
		
	\end{enumerate}
	\end{enumerate}
\end{enumerate}

After each word is represented as a \textit{BinaryArray}, if the query contains the key words \textit{AND}, \textit{OR} or \textit{NOT}, then the corresponding binary operation is performed between the two arrays, based on the precedence of operations previously defined in the grammar of the \textit{CranParser} that tokenizes the query.

The resulting array from performing the binary operations, will be the result of the query. The positions that are marked with a 1 are the number of the documents that are the result of the user's query, being the indexes that are considered relevant for the search.

%----------------------------------------------------------------------------------------
%	PROBLEM 2
%----------------------------------------------------------------------------------------

\section{Conclusions}

Through out the project, the steps for the creation of a retrieval model were identified and developed. Starting from the parsing of the collections, to the implementation of both an \textit{Inverted Index} and a \textit{Boolean Retrieval Model}.

In the same way, the impact of the data structures and the importance that they have within the development of a retrieval model were observed. Especially in both effectiveness and efficiency of a retrieval model.

Finally, the methods to parse both the documents of the \textit{Cranfield} collection and the user's queries were analyzed. This was in order to process and obtain information from the sources of the Information Retrieval Model, along with analyzing the implications that they entail.

%----------------------------------------------------------------------------------------

\end{document}
